\section{References}
Consider the following terms in STLC with references and assume the extensions used in each term are available 
\begin{itemize}
  \item $\mathtt{ref}\ \mathtt{True} \coloneq \mathtt{False}$
  \item $0\coloneq \mathtt{Succ}\ \mathtt{Zero}; \mathtt{Pred}\ !(0)$
  \item $\mathtt{let}\ x = \mathtt{ref}\ \mathtt{Zero}\ \mathtt{in}\ \mathtt{Succ}\ (!x)$
  \item $\mathtt{let}\ x = \mathtt{ref}\ (\lambda y:\mathtt{Nat}.\mathtt{Succ}\ y)\ \mathtt{in}\ (!x)\ ((!x)\ \mathtt{Zero})$
\end{itemize}

For each of these terms, perform the following tasks
\begin{enumerate}
  \item Find a store typing such that the term is typable 
  \item Draw a typing derivation for the term
  \item Evaluate the term to a value and give the store at each step of the evaluation
\end{enumerate}

As in TAPL, these terms use the shorthand $t_1;t_2$ for the term $(\lambda x:\mathtt{Unit}.t2)\ t_1$ where $x\notin FV(t2)$
We also assume whenever a new store location needs to be allocated the lowest free store location is used, and there is no garbage collection.

\section{Implementation}
In this task we will implement STLC with references.
You can use one of the templates provided in the forum.\\
\emph{
    The templates already implement some parts, among them capture-avoiding substitution.
    You only have to change the parts marked with a \texttt{TODO}-comment.
}

\begin{enumerate}
  \item Implement an interpreter for STLC with {\texttt{Unit}} and References, according to the rules in TAPL.
  \item Implement a type checker for that language,
    that assumes the store\footnote{and thus, the store typing} to \emph{always} be the empty one.
\end{enumerate}
