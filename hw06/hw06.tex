\section{STLC with Extensions}

Give derivations for the following typing judgements,
if this possible for some substitutions of the $\square$s\footnote{Each $\square$ can be substituted independently.}.
If not, just state that it is not possible.\\
\begin{enumerate}
  \item $\square\vdash \lambda{b:\mathtt{Bool}}. \mathtt{if}\ b\ \mathtt{then}\ (\mathtt{iszero}\ p.1)\ \mathtt{else}\ (\mathtt{iszero}\ p.2) : \square$ 
    \\ Extensions: booleans, pairs (11-5)
  \item $\vdash \lambda{x:\square}. (\mathtt{case}\ x\ \mathtt{of}\ \mathtt{inl}\ y\Rightarrow{\mathtt{iszero}\ y}\mid\mathtt{inr}\ z\Rightarrow{z}) : \square\to\mathtt{Bool}$ \\
    Extensions: booleans, numbers, sums (11-9) 
  \item   $\vdash \lambda{x:\mathtt{A}}. \square : \mathtt{A} \to (\mathtt{A}+(\mathtt{A}\times\mathtt{A})) $\\
    Extensions: base types (11-1), pairs (11-5), sums(11-9) 
  \item $\vdash \mathtt{case}\ \{a=\mathtt{true}, b=\mathtt{false}\}\ \mathtt{of}<a=x>\Rightarrow x : \square $\\
    Extensions: booleans, records(11-7), variants(11-11)
  \item   $\square \vdash \mathtt{case}\ p\ \mathtt{of}<\hspace{-0.2em}c=x\hspace{-0.1em}>\Rightarrow\ \mathtt{succ}\ x : \mathtt{Nat} $ \\
    Extensions: numbers, variants(11-11) 
\end{enumerate}

\section{The \texttt{Option} type}
We want to extend STLC with $\mathtt{Option}\ t$
\footnote{similar to $\mathtt{Maybe}\ t$ in Haskell~(with cases $\mathtt{Nothing}$ and $\mathtt{Just}\ t$) or $\mathtt{optional<}t\mathtt{>}$ in C++.
} \emph{directly} (i.e., not via sum types). To do so, we first extend the syntax as follows:\\
\begin{minipage}{\textwidth}
  \begin{minipage}{0.45\textwidth}
    \[
      \begin{array}{rclr}
        t &\coloneq& \cdots &\emph{terms} \\
        &&\mathtt{None}&\textit{} \\
        &&\mathtt{Some}~t&\textit{} \\
        &&\mathtt{case}~t~\mathtt{of}~\mathtt{None}\Rightarrow t\mid\mathtt{Some}~x\Rightarrow t \\
        \phantom{\quad}\\
        \phantom{\quad}\\
      \end{array}
    \]
  \end{minipage}
  \hfill
  \begin{minipage}{0.45\textwidth}
    \[
      \begin{array}{rclr}
        v &\coloneq& \cdots&\textit{values:} \\
        &&\mathtt{None}&\textit{} \\
        &&\mathtt{Some}~v&\textit{} \\
        \\
        T &\coloneq& \cdots&\textit{types:} \\
        &&\mathtt{Option}~T&\textit{}
      \end{array}
    \]
  \end{minipage}
  \hfill
  \vspace{1em}
\end{minipage}

\begin{enumerate}
  \item Give small-step evaluation semantics for the extension of STLC with $\mathtt{Option}$.
  \item Give appropriate typing rules.
  \item Prove soundness (progress and preservation) for STLC with $\mathtt{Option}$ and pairs (11-5).
\end{enumerate}

\section{Implementation}
This task is an implementation task. 
You can use a programming language of your choice.
However, we recommend that you use a language that 
\begin{itemize}
  \item allows you to easily define and
    pattern match on data types or a similar construct (also called `\verb|enum|', or `tagged/discriminated union' in some languages), and
  \item allows you to define recursive functions on them
\end{itemize}
E.g.,
Haskell,
Scala\footnote{Use case classes or enums (Scala 3)!},
most dialects of ML, 
Rust\footnote{Note that manual memory management may be an additional challenge here!},
many LISP dialects\footnote{Here, you can use sexps to represent terms.},
$\dots$.

\begin{enumerate}
  \item Implement a typechecker for simply typed lambda-calculus, extended with $\mathtt{unit}$.\\
    You do \emph{not} have to implement a parser or interpreter. Just define data types for terms and types and directly call the function
    with some example values for testing.\\
    \emph{Hint: For variables you can simply use the $\mathtt{String}$ data type in your chosen language}
  \item Extend your typechecker with at least one of the extensions described in Chapter 11 of TAPL.
\end{enumerate}

\section{Substructural type systems}
In the lecture on STLC, we made some choices about typing contexts:
\begin{itemize}
  \item Each variable can only occur at most once in each environment (via well-formedness).
  \item Bindings in the typing context can be reodered
  \item Bindings cann be added to the typing context
\end{itemize}
All of those are called \emph{structural} properties.\\
We will now consider a different set of choices we can make, leading to the following calculus
\footnote{Inspired by \emph{Linear Haskell: practical linearity in a higher-order polymorphic language} \url{https://doi.org/10.1145/3158093}}
\begin{minipage}{\textwidth}
  \begin{minipage}{0.45\textwidth}
    \[
      \begin{array}{lccr}
        t & \coloneq & & \emph{Terms}\\
        & & x & \emph{Variables}\\
        && t_1\ t_2 & \emph{Applications}\\
        && \lambda_1 x:T.t & \emph{Abstractions}\\
        && (t, t) & \emph{Pairs}\\
        && \mathtt{case}\ t \ \mathtt{of}\ (x,x) \Rightarrow t & \emph{Deconstruction}\\
        \\
        v & \coloneq & & \emph{Values}\\
        & & \lambda_1 x:T.t & \emph{Function Values}\\
        && (v,v) & \emph{Pair Values}\\
        \\
      \end{array}
    \]
  \end{minipage}
  \hfill
  \begin{minipage}{0.45\textwidth}
    \[
      \begin{array}{lccr}
        T & \coloneq && \emph{Types}\\
        && T\rightarrow_1 T & \emph{(Linear) Function Type}\\
        && T \times T & \emph{Product Type}\\
        \\
        \Gamma & & \emph{Contexts}\\
        & & \emptyset & \emph{Empty Contexts}\\
        & & \Gamma, x :_1 : T &\emph{(linear) binding}\\
        \phantom{\quad}\\
        \phantom{\quad}\\
        \phantom{\quad}\\
        \phantom{\quad}
      \end{array}
    \]
  \end{minipage}
  \hfill
  \vspace{0.5em}
\end{minipage}
\center{\textbf{Context Splitting}}\quad
$\Gamma = \Gamma_1\circ\Gamma_2$\\
\vspace{1em}
\begin{minipage}{0.2\textwidth}
  \begin{prooftree}
    \AxiomC{\quad}
    \UnaryInfC{$\emptyset = \emptyset \circ \emptyset $}
  \end{prooftree}
\end{minipage}
\hfill
\begin{minipage}{0.35\textwidth}
  \begin{prooftree}
    \AxiomC{$\Gamma = \Gamma_1\circ\Gamma_1$}
    \UnaryInfC{$\Gamma,x:_1 T = (\Gamma_1,x:_1 T) \circ \Gamma_2$}
  \end{prooftree}
\end{minipage}
\hfill
\begin{minipage}{0.35\textwidth}
  \begin{prooftree}
    \AxiomC{$\Gamma= \Gamma_1\circ\Gamma_2$}
    \UnaryInfC{$\Gamma,x:_1 T = \Gamma_1 \circ (\Gamma_2,x:_1 T)$}
  \end{prooftree}
\end{minipage}
\hfill
\vspace{1em}
\center{\textbf{Typing Rules}}\quad $\Gamma\vdash t:T$\\
\vspace{1em}
\begin{minipage}{\textwidth}
  \begin{minipage}{0.2\textwidth}
    \begin{prooftree}
      \AxiomC{\quad}
      \RightLabel{T-Var}
      \UnaryInfC{$x:_1 T \vdash x:T$}
    \end{prooftree}
  \end{minipage}
  \hfill
  \begin{minipage}{0.75\textwidth}
    \begin{prooftree}
      \AxiomC{$\Gamma_1\vdash t_1:T_2\rightarrow_1 T$}
      \AxiomC{$\Gamma_2\vdash t_2:T_2$}
      \AxiomC{$\Gamma = \Gamma_1\circ\Gamma_2$}
      \RightLabel{\textsc{T-App}}
      \TrinaryInfC{$\Gamma \vdash t_1\ t_2 :T$}
    \end{prooftree}
    \hfill
  \end{minipage} 
 \hfill
  \vspace{1em}
\end{minipage}

\begin{minipage}{\textwidth}
  \begin{minipage}{0.35\textwidth}
    \begin{prooftree}
      \AxiomC{$\Gamma = \Gamma_1\circ\Gamma_2$}
      \AxiomC{$\Gamma_1 \vdash t:T$}
      \RightLabel{\textsc{T-Ctr}}
      \BinaryInfC{$\Gamma \vdash t:T$}
    \end{prooftree}
    \hfill
  \end{minipage}
  \hfill
  \begin{minipage}{0.5\textwidth}
    \begin{prooftree}
      \AxiomC{$\Gamma x:_1 T_1 \vdash t:T_2$}
      \AxiomC{$x\notin\mathrm{dom}(\Gamma)$}
      \RightLabel{\textsc{T-Abs}}
      \BinaryInfC{$\Gamma\vdash \lambda_a x:T_1.t : T_2\rightarrow_1 T$}
    \end{prooftree}
  \end{minipage}
  \vspace{1em}
\end{minipage}

\begin{prooftree}
  \AxiomC{$\Gamma = \Gamma_1\circ\Gamma_2$}
  \AxiomC{$\Gamma_1\vdash t_1:T_{21}\times T_{22}$}
  \AxiomC{$\Gamma_2, x_1:_1 T_{21},x_2:_1 T_{22}\vdash t_2:T$}
  \RightLabel{\textsc{T-Match}}
  \TrinaryInfC{$\Gamma\vdash\mathtt{case}\ t_1\ \mathtt{of}\ (x_1,x_2)\Rightarrow t_2 :T$}
\end{prooftree}
\begin{prooftree}
  \AxiomC{$\Gamma = \Gamma_1\circ\Gamma_2$}
  \AxiomC{$\Gamma_1\vdash t_1:T_1$}
  \AxiomC{$\Gamma_2\vdash t_2:T_2$}
  \RightLabel{\textsc{T-Pair}}
  \TrinaryInfC{$\Gamma\vdash (t_1,t_2) : T_1 \times T_2$}
\end{prooftree}
Some remarks:
\begin{itemize}
  \item For simplicity, assume that the evaluation rules for $\rightarrow_1$ are like for $\rightarrow$ in STLC.
  \item Assume that (in addition to the constructions there), we have some set of base types $A$, $B$, $\dots$.
  \item Note that, although we write \(\Gamma = \Gamma_1\circ\Gamma_2\), the order in \(\Gamma\) is \emph{not}
    uniquely determined by \(\Gamma_1\) and \(\Gamma_2\).
\end{itemize}


\begin{enumerate}
  \item Show that \emph{Exchange} still holds, i.e. prove: \\
    If $\Gamma_a,x_a:_1 T_a, x_b :_1 T_b, \Gamma_b\vdash t:T$, then 
    $\Gamma_a, x_b:_1 T_b, x_a:_a T_a, \Gamma_b \vdash t:T$
  \item Show that \emph{Weakening} does not\\
    That is, find $\Gamma$, $x$, $t$, $T$ and $T^{\prime}$ such that 
    $\Gamma\vdash t:T$ and $x\notin\mathrm{dom}(\Gamma)$ but
    $\Gamma, x:_1 T^{\prime}\vdash t:T$ does \emph{not} hold
  \item What do those properties mean for programming in a language based on this calculus?
  \item Can you think of a situation where this might be useful
  \item Can we add extensions to recover the features we lost on the term level?\\
    That is, define an extension to the syntax of terms along with evaluation and typing rules.
    Then show (informally) how to use those
\end{enumerate}

\section{Types and Programming Languages}
Read trough chapter 11 of Types and Programming Languages and post your questions in the dedicated forum thread.
