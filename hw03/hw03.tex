\section*{Task 1: Evaluation of Untyped Lambda Calculus}

Reduce the following terms in the lambda calculus until they reach a normal form.
Use the call-by-name evaluation strategy.

\begin{enumerate}
  \item $(\lambda\ x.\lambda\ y.y)\ (\lambda\ a.a\ a)$ 
    % (\x.\y.y) (\a.a a) --> \y.y
  \item $((\lambda\ x. \lambda\ y.x\ y)\ (\lambda\ z.\lambda\ w.w))\ (\lambda\ a.a\ a) $
    % (\x.\y.x y) (\z.\w.w) (\a.a a) --> (\y.(\z.\w.w) y) (\a.a a) --> (\z.\w.w) (\a.a a) --> \w.w
\end{enumerate}

You do not have to draw a derivation tree for the reduction relation, but you have to write down all
reduction steps. So, e.g., for $\lambda\ f.f;(\lambda\ x.x);(\lambda\ x.x)$
you would write down the following:
\begin{align*}
  & (\lambda\ f.f\ (\lambda\ x.x))\ (\lambda\ x.x) \\
  \mapsto &  (\lambda\ x.x)\ (\lambda\ x.x)\\
  \mapsto & (\lambda\ x.x)
\end{align*}

\section*{Task 2: Church encoding}
In the book, natural numbers were encoded in the lambda calculus as follows:
\begin{align*}
  c_0 & = & \lambda\ s. \lambda\ z.z\\
  c_1 & = & \lambda\ s. \lambda\ z.s\ z\\
  \vdots\\
  c_n & = & \lambda\ s. \lambda z. s\ s\ \dots\ z; & \text{ (with $s$ appearing $n$ times)}
\end{align*}

Similarly, we are now going to encode binary trees in lambda calculus as folds.
That is, the tree
\begin{center}
  \tikz{
    \node[draw,circle]{$v$} child {
      node[draw,circle]{$v_1$} child { node[draw]{$v_{11}$} } child { node[draw]{$v_{12}$} }
    } child {node[draw]{$v_2$}};
  }
\end{center}
is encoded as
\begin{align*}
  \lambda\ f.\lambda\ g.f\ v\ (f\ v_1\ (g\ v_{11})\ (g\ v_{12}))\ (g\ v_2)
\end{align*}

The basic tree functions look like this:
\begin{align*}
  \mathtt{leaf} & \coloneq & \lambda\ v.\lambda\ f.\lambda\ g.g\ v \\
  \mathtt{node} & \coloneq & \lambda\ v.\lambda\ l.\lambda\ r.\lambda\ f.\lambda\ g.f\ v\ (l\ f\ g)\ (r\ f\ g) \\
  \mathrm{fold} & \coloneq & \lambda\ f.\lambda\ g.\lambda\ t.t\ f\ g
\end{align*}

Write the following functions, operating on trees:
\begin{enumerate}
  \item $\mathtt{sum_leaves}$: sums the values in the \emph{leaves}.
  \item $\mathtt{map}$: given a function $f$, returns a tree with the same structure but $f$ applied to the value in every inner node and leaf.
  \item $\mathtt{size}$: returns the number of nodes
\end{enumerate}

You may use the functions defined in the book, such as $\mathtt{plus}$, $c_0,c_1,\dots$

\section*{Task 3: Call-by-name lambda calculus}

In the lecture, you have seen the rules for the call-by-value (CBV) lambda calculus.

For call-by-name (CBN), we want to apply a function \emph{before} evaluating the arguments, e.g.,
\begin{align*}
  & (\lambda\ x.x)\ ((\lambda\ y.y)\ (\lambda\ z.z\ z)) \\
  \mapsto & (\lambda\ y.y)\ (\lambda\ z.z\ z)\\
  \mapsto & (\lambda\ z.z\ z)
\end{align*}

\begin{itemize}
  \item Give inference rules defining the (one-step) evaluation relation for the call-by-name lambda calculus
  \item Evaluate the following term using both call-by-value and call-by-name. As in task 1, you do not need to draw the derivation tree for each step
  \item What are the possible up- and downsides of using cbv and cbn?
\end{itemize}

\begin{align*}
    (\lambda\ f.\lambda\ v.v\ v)\ ((\lambda\ x.x)\ (\lambda\ y.\lambda\ z.y))\ ((\lambda\ a.a\ a)\ (\lambda\ y.y))
\end{align*}
%% λ mapM_ print $ steps cbn ex1 -- 7 steps
%% (((\f. (\v. (v v))) ((\x. x) (\y. (\z. y)))) ((\a. (a a)) (\y. y)))
%% ((\v. (v v)) ((\a. (a a)) (\y. y)))
%% (((\a. (a a)) (\y. y)) ((\a. (a a)) (\y. y)))
%% (((\y. y) (\y. y)) ((\a. (a a)) (\y. y)))
%% ((\y. y) ((\a. (a a)) (\y. y)))
%% ((\a. (a a)) (\y. y))
%% ((\y. y) (\y. y))
%% (\y. y)
%% λ mapM_ print $ steps cbv ex1 -- 6 steps
%% (((\f. (\v. (v v))) ((\x. x) (\y. (\z. y)))) ((\a. (a a)) (\y. y)))
%% (((\f. (\v. (v v))) (\y. (\z. y))) ((\a. (a a)) (\y. y)))
%% ((\v. (v v)) ((\a. (a a)) (\y. y)))
%% ((\v. (v v)) ((\y. y) (\y. y)))
%% ((\v. (v v)) (\y. y))
%% ((\y. y) (\y. y))
%% (\y. y)
%

\section*{Task 4}
Read chapter 5 of Types and Programming languages and post your questions in the dedicated forum thread.
Optionally also watch the talk \href{https://www.youtube.com/watch?v=dCuZkaaou0Q}{``It's Time for a New Old Language''} by Guy Steele and post your questions in the forum.
