\section{Subtyping}
Assume the following syntax for types
\[
  \begin{array}{lcr}
    \mathbb{B} &::= A \;|\; B \;|\; C \;|\; D \;|\; E \;|\; F &\textit{{\color{black}(base types)}}\\
    T &::= \top{} \;|\; \bot{} \;|\; \{{l_i: T_i}_{i\in 1..{n}}\} \;|\; T \to T \;|\; \mathbb{B}
  \end{array}
\]
We use $\top$ for \texttt{Top} and $\bot$ for \texttt{Bottom}.

Additionally assume that the subtyping relation is defined by the following rules given in the lecture:\\
\begin{itemize}
  \item \textsc{S-RcdWidth}, \textsc{S-RcdPerm}, \textsc{S-RcdDepth}
  \item \textsc{S-Arrow}
  \item \textsc{S-Top}, \textsc{S-Bottom}
  \item \textsc{S-Refl}, \textsc{S-Trans}
\end{itemize}
Along with the following subtyping relationships between base types 
\[
  \begin{array}{lcr}
    D<:B & D<:C & B<:A\\
    C<:A & E<:D & E<:F
  \end{array}
\]

Show the following subtyping judgements by drawing a derivation tree:
\begin{enumerate}
  \item $\{x : C\} \to C \;<:\; \{x: D, y: B\} \to A$
  \item $ (B\to C) \to D  \;<:\;  (A \to D) \to B$
  \item $ (E \to \bot) \to \bot \;<:\; (F \to \bot) \to \bot$
\end{enumerate}

For the following pairs of types, what is their \emph{join} and what is their \emph{meet}?
Look at the definition of \emph{join} and \emph{meet} in TAPL, that is, at Definition 16.3.1.
\begin{enumerate}
  \item $B$ and $C$
  \item $F$ and $B$
  \item $\{x: C, y: \top\}$ and $\{x: B, z: \bot\}$
\end{enumerate}

\section{Reflexivity and Transitivity of Subtyping}
Assume that the subtyping relation is defined by the rules\\
\begin{minipage}{\textwidth}
  \vspace{1em}
  \begin{minipage}{0.4\textwidth}
    \begin{prooftree}
      \AxiomC{}
      \RightLabel{\textsc{S-Base-Refl}}
      \UnaryInfC{$\mathbb{B} <: \mathbb{B}$}
    \end{prooftree}
    \hfill
  \end{minipage}
  \begin{minipage}{0.5\textwidth}
    \begin{prooftree}
      \AxiomC{$\mathbb{B}_1<:\mathbb{B}_2$}
      \AxiomC{$\mathbb{B}_2<:\mathbb{B}_3$}
      \RightLabel{\textsc{S-Base-Trans}}
      \BinaryInfC{$\mathbb{B}_1 <: \mathbb{B}_3$}
    \end{prooftree}
    \hfill
  \end{minipage}
  \hfill
\end{minipage}
\vspace{1em}
\begin{minipage}{\textwidth}
  \begin{minipage}{0.45\textwidth}
    \begin{prooftree}
      \AxiomC{$T_1<:S_1$}
      \AxiomC{$S_2<:T_2$}
      \RightLabel{\textsc{S-Arrow}}
      \BinaryInfC{$S_1\rightarrow S_2 <: T_1<:T_2$}
    \end{prooftree}
    \hfill
  \end{minipage}
  \begin{minipage}{0.2\textwidth}
    \begin{prooftree}
      \AxiomC{}
      \RightLabel{\textsc{S-Top}}
      \UnaryInfC{$T <: \top$}
    \end{prooftree}
    \hfill
  \end{minipage}
  \begin{minipage}{0.2\textwidth}
    \begin{prooftree}
      \AxiomC{}
      \RightLabel{\textsc{S-Bottom}}
      \UnaryInfC{$\bot <: T$}
    \end{prooftree}
    \hfill
  \end{minipage}
\end{minipage}
in addition to the ones \emph{for base types} defined above.
The syntax of types is:
\[
  \begin{array}{lr}
    T &::= \top{} \;|\; \bot{} \;|\; T \to T \;|\; \mathbb{B}
  \end{array}
\]
with \(\mathbb{B}\) as above.

\begin{itemize}
  \item Show that Reflexivity holds for the subtyping relation\footnote{That is, \textsc{S-Refl} is \emph{admissible}.},
    i.e., show that we have $T<:T$ for all types $T$.
  \item  Show that Transitivity holds for the subtyping relation\footnote{That is, \textsc{S-Trans} is \emph{admissible}.},
    i.e., show that (for all types $T_1$, $T_2$, $T_3$), if $T_1 <: T_2$ and $T_2 <: T_3$, then $T_1 <: T_3$.
\end{itemize}

\section{TAPL}
Read chapters 15 and 16 in TAPL and post your questions in the dedicated forum thread.

