\section{Typing}
Consider the following programming language of integers and floating point numbers 
\[
  \begin{array}{lcrr}
    & & n\in\mathbb{Z} \quad f\in\mathbb{R}\\
    \oplus & \coloneq & + \mid - \mid * \mid / &\emph{Binary Operations}\\
    t & \coloneq & & \emph{Terms} \\
    & & \mathtt{Int}(n) & \emph{Integers}\\ 
    & \mid & \mathtt{Fl}(f) & \emph{Floating Point Numbers} \\
    & \mid & t \oplus t & \emph{Operations}\\
    & \mid & \mathtt{Round}(t) & \emph{Round up/down}\\
    & \mid & \mathtt{AsFloat}(t) & \emph{Convert int to float}\\
    & \mid & \mathtt{pow}(t,t) & \emph{Exponentials}
  \end{array}
\]
In this language there are floating point numbers and integers, tagged with $\mathtt{Fl}$ and $\mathtt{Int}$, respectively.
These tags are added to avoid ambiguity for terms such as $3$ which could be interpreted as either the floating point number $3.0$ or the integer $3$. 
Instead, we use $\mathtt{Int}(3)$ and $\mathtt{Fl}(3)$ to make it clear which one we are referring to.\\
We can take sums, differences, products, and quotients of these numbers, as well as round a value and convert integers to floating point numbers.
Lastly, we can take the exponential of two numbers, e.g. $\mathtt{pow}(\mathtt{Fl}(3),\mathtt{Int}(2))$ refers to the term $3^2$.\\
To this language we add the following type system:
\[
  \begin{array}{lcrr}
    \tau & \coloneq \mathtt{Int} \mid \mathtt{Fl} & \emph{Types}
  \end{array}
\]
\begin{minipage}{\textwidth}
  \begin{minipage}{0.2\textwidth}
    \begin{prooftree}
      \AxiomC{\quad}
      \RightLabel{\textsc{Ty-Int}}
      \UnaryInfC{$\mathtt{Int}(n) : \mathtt{Int}$}
    \end{prooftree}
  \end{minipage}
  \hfill
  \begin{minipage}{0.5\textwidth}
    \begin{prooftree}
      \AxiomC{$t_1:\tau$}
      \AxiomC{$t_2:\tau$}
      \RightLabel{\textsc{Ty-BinOp}}
      \BinaryInfC{$t_1\oplus t_2 : \tau$}
    \end{prooftree}
  \end{minipage}
  \hfill
  \begin{minipage}{0.2\textwidth}
    \begin{prooftree}
      \AxiomC{\quad}
      \RightLabel{\textsc{Ty-Float}}
      \UnaryInfC{$\mathtt{Fl}(n) : \mathtt{Fl}$}
    \end{prooftree}
  \end{minipage}
  \hfill
  \vspace{1em}
  \begin{minipage}{0.3\textwidth}
    \begin{prooftree}
      \AxiomC{$t:\mathtt{Fl}$}
      \RightLabel{\textsc{Ty-Round}}
      \UnaryInfC{$\mathtt{Round}(t):\mathtt{Int}$}
    \end{prooftree}
  \end{minipage}
  \hfill
  \begin{minipage}{0.3\textwidth}
    \begin{prooftree}
      \AxiomC{$t:\mathtt{Int}$}
      \RightLabel{\textsc{Ty-ToFloat}}
      \UnaryInfC{$\mathtt{AsFloat}(t):\mathtt{Fl}$}
    \end{prooftree}
  \end{minipage}
  \hfill
  \begin{minipage}{0.3\textwidth}
    \begin{prooftree}
      \AxiomC{$t_1:\tau$}
      \AxiomC{$t_2:\mathtt{Int}$}
      \RightLabel{\textsc{Ty-Pow}}
      \BinaryInfC{$\mathtt{pow}(t_1,t_2):\mathtt{Fl}$}
    \end{prooftree}
  \end{minipage}
  \hfill
  \vspace{1em}
\end{minipage}
Now consider the following terms in this language
\begin{enumerate}
  \item $(\mathtt{Fl}(1) + \mathtt{pow}(\mathtt{Fl}(5),\mathtt{Fl}(0.5)))/\mathtt{Fl}(2)$
  \item $\mathtt{round}(\mathtt{Fl}(3) * \mathtt{AsFloat}(\mathtt{Int}(4)))$
  \item $\mathtt{pow}(\mathtt{AsFloat}(\mathtt{Int}(2)),\mathtt{Int}(64))$
  \item $\mathtt{Fl}(4) / \mathtt{Int}(2)$
\end{enumerate}
For each of these terms, state if they can be typed using the rules above.
If they can be, give a typing derivation, otherwise give a reason why the term is not well-typed.
Here is an example tree for the term $((\mathtt{AsFloat}(\mathtt{Int}(4))*\mathtt{Fl}(3.14))*\mathtt{pow}(\mathtt{Int}(2),\mathtt{Int}(3)))/\mathtt{Fl}(3)$\\
\begin{minipage}{\textwidth}
  \vspace{1em}
  \begin{prooftree}
    \AxiomC{}
    \UnaryInfC{$\mathtt{Int}(4):\mathtt{Int}$}
    \UnaryInfC{$\mathtt{AsFloat}(\mathtt{Int}(4)):\mathtt{Fl}$}
    \AxiomC{}
    \UnaryInfC{$\mathtt{Fl}(3.14):\mathtt{Fl}$}
    \BinaryInfC{$\mathtt{AsFloat}(\mathtt{Int}(4))*\mathtt{Fl}(3.14):\mathtt{Fl}$}

    \AxiomC{}
    \UnaryInfC{$\mathtt{Int}(2):\mathtt{Int}$}
    \AxiomC{}
    \UnaryInfC{$\mathtt{Int}(3):\mathtt{Int}$}
    \BinaryInfC{$\mathtt{pow}(\mathtt{Int}(2),\mathtt{Int}(3)) : \mathtt{Fl}$}
    \BinaryInfC{$(\mathtt{AsFloat}(\mathtt{Int}(4))*\mathtt{Fl}(3.14))*\mathtt{pow}(\mathtt{Int}(2),\mathtt{Int}(3)) : \mathtt{Fl}$}

    \AxiomC{}
    \UnaryInfC{$\mathtt{Fl}(3):\mathtt{Fl}$}
    \BinaryInfC{$((\mathtt{AsFloat}(\mathtt{Int}(4))*\mathtt{Fl}(3.14))*\mathtt{pow}(\mathtt{Int}(2),\mathtt{Int}(3)))/\mathtt{Fl}(3)$}
  \end{prooftree}
\end{minipage}

\section{Progress and Preservation}
Given the language above, we extend it with the following operational semantics:
\[
  \begin{array}{lcrr}
    v & \coloneq & \mathtt{Fl}(f) \mid \mathtt{Int}(n) & \emph{Values}\\
  \end{array}
\]
\begin{align*}
  \mathtt{Fl}(f_1) \oplus \mathtt{Fl}(f_2) &\mapsto \mathtt{Fl}(f_1\oplus f_2) & \text{\textsc{E-FlOp}}\\
  \mathtt{Int}(i_1) \oplus \mathtt{Int}(i_2) &\mapsto \mathtt{Int}(i_1\oplus i_2) & \text{\textsc{E-IntOp}}\\
  \mathtt{round}(\mathtt{Fl}(f)) &\mapsto \mathtt{Int}(\lfloor f \rceil) & \text{\textsc{E-RoundFl}}\\
  \mathtt{AsFloat}(\mathtt{Int}(i)) &\mapsto \mathtt{Fl}(i) & \text{\textsc{E-IntFl}}\\
  \mathtt{pow}(\mathtt{Fl}(f),\mathtt{Int}(i)) &\mapsto \mathtt{Fl}(f^i) & \text{\textsc{E-PowFl}}\\
  \mathtt{pow}(\mathtt{Int}(i_1),\mathtt{Int}(i_2)) &\mapsto \mathtt{Fl}(i_1^{i_2}) & \text{\textsc{E-PowInt}}\\
\end{align*}

\begin{minipage}{\textwidth}
  \begin{minipage}{0.45\textwidth}
    \begin{prooftree}
      \AxiomC{$t_1\mapsto t_1^{\prime}$}
      \RightLabel{\textsc{E-BinOp1}}
      \UnaryInfC{$t_1\oplus t_2 \mapsto t_1^{\prime}\oplus t_2$}
    \end{prooftree}
  \end{minipage}
  \hfill
  \begin{minipage}{0.45\textwidth}
    \begin{prooftree}
      \AxiomC{$t_2\mapsto t_2^{\prime}$}
      \RightLabel{\textsc{E-BinOp2}}
      \UnaryInfC{$v\oplus t_2 \mapsto v\oplus t_2^{\prime}$}
    \end{prooftree}
  \end{minipage}
  \hfill
  \vspace{1em}
  \begin{minipage}{0.45\textwidth}
    \begin{prooftree}
      \AxiomC{$t\mapsto t^{\prime}$}
      \RightLabel{\textsc{E-Round}}
      \UnaryInfC{$\mathtt{round}(t) \mapsto \mathtt{round}(t^{\prime})$}
    \end{prooftree}
  \end{minipage}
  \hfill
  \begin{minipage}{0.45\textwidth}
    \begin{prooftree}
      \AxiomC{$t\mapsto t^{\prime}$}
      \RightLabel{\textsc{E-AsFloat}}
      \UnaryInfC{$\mathtt{AsFloat}(t) \mapsto \mathtt{AsFloat}(t^{\prime})$}
    \end{prooftree}
  \end{minipage}
  \hfill
  \vspace{1em}
  \begin{minipage}{0.45\textwidth}
    \begin{prooftree}
      \AxiomC{$t_1\mapsto t_1^{\prime}$}
      \RightLabel{E-Pow1}
      \UnaryInfC{$\mathtt{pow}(t_1,t_2)\mapsto \mathtt{pow}(t_1^{\prime},t_2)$}
    \end{prooftree}
  \end{minipage}
  \hfill
  \begin{minipage}{0.45\textwidth}
    \begin{prooftree}
      \AxiomC{$t_2\mapsto t_2^{\prime}$}
      \RightLabel{E-Pow2}
      \UnaryInfC{$\mathtt{pow}(v,t_2)\mapsto \mathtt{pow}(v,t_2^{\prime})$}
    \end{prooftree}
  \end{minipage}
  \hfill
  \vspace{1em}
\end{minipage}
In the rule \textsc{E-RoundFl}, we use the notation $\lfloor f\rceil$ to denote rounding the number $f$ as usual, i.e. $\lfloor 0.5 \rceil = 1$ and $\lfloor 0.4 \rceil = 0$.\\
Using these semantics, prove \emph{either} \emph{Preservation} or \emph{Progress}. 
Optionally, you can also prove both theorems.
\begin{theorem}\emph{Preservation}\\
  Let $t_1,t_2$ be terms in the language such that $t_1:\tau$ and $t_1\mapsto t_2$. 
  Then $t_2:\tau$
\end{theorem}

\begin{theorem}\emph{Progress}\\
  Let $t_1$ be a wel-typed term, i.e. $t_1:\tau$ for some type $\tau$. 
  Then either $t_1$ is a value, or there is some $t_2$ such that $t_1\mapsto t_2$.
\end{theorem}

\section{Designing a type system}
Take the following syntax for terms and values, where $n$ are integers with operational semantics below
\[
  \begin{array}{lcrr}
    t & \coloneq & &\emph{Terms}\\
      &          & n \\
      &\mid      & \mathtt{Nil} \\
      &\mid      & \mathtt{Cons}(t,t) \\
      &\mid      & \mathtt{tail}(t)\\
      &\mid      & \mathtt{succ}(t)\\
      &          & \mathtt{pred}(t)\\
      \\
    lv & \coloneq & \mathtt{Nil}\mid \mathtt{Cons}(n,lv) & \emph{List values}\\
    v & \coloneq & n\mid lv& \emph{Values}\\
  \end{array}
\]
\begin{align*}
  \mathtt{tail}(\mathtt{Cons}(n,lv)) & \mapsto lv\\
  \mathtt{tail}(\mathtt{Nil}) & \mapsto \mathtt{Nil}\\
  \mathtt{succ}(n) & \mapsto n+1\\
  \mathtt{pred}(n) & \mapsto n-1\\ 
\end{align*}
\begin{minipage}{\textwidth}
  \begin{minipage}{0.45\textwidth}
    \begin{prooftree}
      \AxiomC{$t_1\mapsto t_1^{\prime}$}
      \RightLabel{\textsc{E-Cons1}}
      \UnaryInfC{$\mathtt{Cons}(t_1,t_2)\mapsto \mathtt{Cons}(t_1^{\prime},t_2)$}
    \end{prooftree}
  \end{minipage}
  \hfill
  \begin{minipage}{0.45\textwidth}
    \begin{prooftree}
      \AxiomC{$t_2\mapsto t_2^{\prime}$}
      \RightLabel{\textsc{E-Cons2}}
      \UnaryInfC{$\mathtt{Cons}(n,t_2)\mapsto \mathtt{Cons}(n,t_2^{\prime})$}
    \end{prooftree}
  \end{minipage}
  \begin{minipage}{0.3\textwidth}
    \begin{prooftree}
      \AxiomC{$t\mapsto t^{\prime}$}
      \RightLabel{\textsc{E-Tail}}
      \UnaryInfC{$\mathtt{tail}(t) \mapsto \mathtt{tail}(t^{\prime})$}
    \end{prooftree}
  \end{minipage}
  \hfill
  \begin{minipage}{0.3\textwidth}
    \begin{prooftree}
      \AxiomC{$t\mapsto t^{\prime}$}
      \RightLabel{\textsc{E-Succ}}
      \UnaryInfC{$\mathtt{succ}(t) \mapsto \mathtt{succ}(t^{\prime})$}
    \end{prooftree}
  \end{minipage}
  \hfill
  \begin{minipage}{0.3\textwidth}
    \begin{prooftree}
      \AxiomC{$t\mapsto t^{\prime}$}
      \RightLabel{\textsc{E-Pred}}
      \UnaryInfC{$\mathtt{pred}(t) \mapsto \mathtt{pred}(t^{\prime})$}
    \end{prooftree}
  \end{minipage}
  \hfill
  \vspace{1em}
\end{minipage}
Design a type system for this language using the following steps
\begin{enumerate}
  \item Give a possible syntax of types
  \item Give typing rules for this language (stuck terms should not be well typed)\\
    \emph{(Hint: $\mathtt{Cons}(1,\mathtt{Nil})$ is a value, but $\mathtt{Cons}(\mathtt{Nil},\mathtt{Nil})$ is not.)}
  \item Show \emph{Progress} and \emph{Preservation} for the type system you designed
\end{enumerate}

\section{Types and Programming Languages}
Read through chapter 7 of ``Types and Programming Languages'' and optionally also chapter 6 and post your questions in the dedicated forum thrad.

\section{Optional: Nameless Representations}
Transform the following terms into the nameless representation introduced in chapter 6 of types and programming languages 
\begin{enumerate}
  \item $\lambda\ x. \lambda y. x\ y$
  \item $\lambda\ x. x\ (\lambda\ y. \lambda\ z. y)$
  \item $\lambda\ x. ((\lambda y. x) (\lambda z. z))$
\end{enumerate}
