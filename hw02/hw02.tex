\section*{Evaluation Contexts}
Consider the grammar from homework 01, with $+,-,*$ shortened to $\oplus$:
\[
  \begin{array}{l c r r}
    \oplus & \coloneq & + \mid - \mid * & \emph{Binary Operations}\\
    t & \coloneq & n \mid t \oplus t  \mid \mathtt{ifz}(t;t,t) & \emph{Terms}\\
  \end{array}
\]
We define the following operational semantics for this language

\begin{align*}
  v & \coloneq n & \emph{Values}\\
  v_1 \oplus v_2 & \rightarrow & (v_1\oplus v_2)\\
  \mathtt{ifz}(0;t_1,t_2) & \rightarrow &  t_1\\
  \mathtt{ifz}(v;t_2,t_2) & \rightarrow & t_2 \text{if } n\neq 0\\
\end{align*}

\begin{minipage}{\textwidth}
  \begin{minipage}{0.3\textwidth}
    \begin{prooftree}
      \AxiomC{$t_1\rightarrow t_1^{\prime}$}
      \UnaryInfC{$t_1\oplus t_2 \rightarrow t_1^{\prime}\oplus t_2$}
    \end{prooftree}
  \end{minipage}
  \hfill
  \begin{minipage}{0.3\textwidth}
    \begin{prooftree}
      \AxiomC{$t_2\rightarrow t_2^{\prime}$}
      \UnaryInfC{$v \oplus t_2 \rightarrow v \oplus t_2^{\prime}$}
    \end{prooftree}
  \end{minipage}
  \hfill
  \begin{minipage}{0.3\textwidth}
    \begin{prooftree}
      \AxiomC{$t_1\rightarrow t_1^{\prime}$}
      \UnaryInfC{$\mathtt{ifz}(t_1;t_2,t_3) \rightarrow \mathtt{ifz}(t_1^{\prime};t_2,t_3)$}
    \end{prooftree}
  \end{minipage}
  \hfill
  \vspace{2em}
\end{minipage}

Here $(v_1 + v_2)$ means we take the integer given by the sum of values $v_1$ and $v_2$ (instead of the term $v_1+v_2$, and similarly for $-$ and $*$.
\begin{itemize}
  \item Which of the above rules are congruence rules? Which are computation rules?
  \item Re-formulate the congruence rules using \emph{evaluation contexts}.
    That is, write an abstract grammar for $E$ such that all congruence rules above can be replaced with the single rule:
\end{itemize}

\begin{minipage}{\textwidth}
  \begin{prooftree}
    \AxiomC{$t\rightarrow t^{\prime}$}
    \UnaryInfC{$E[t] \rightarrow E[t^{\prime}]$}
  \end{prooftree}
  \hfill
  \vspace{1em}
\end{minipage}

Here, $E[t]$ denotes the result of ``plugging in'' $t$ in $E$, i.e., replacing $\square$ with $t$ in $E$.

\section*{Evaluation}

\subsection*{Derivation Trees}

Using the above language and the given evaluation rules, give a derivation tree for evaluating the following term by one step
\begin{align*}
  \mathtt{ifz}((3*4)-5;(7*7)+15,4+4)
\end{align*}
The tree should begin like this \\
\begin{minipage}{\textwidth}
  \begin{prooftree}
    \AxiomC{\dots}
    \UnaryInfC{$\mathtt{ifz}((3*4)-5;(7*7)+15,4+4) \mapsto \mathtt{ifz}(12-5;(7*7)+15,4+4)$}
  \end{prooftree}
  \hfill
  \vspace{1em}
\end{minipage}
If you are using LaTeX for your solution, use the package ``bussproofs'' for the derivation tree.

\subsection*{Normal Forms}

\begin{itemize}
  \item Give an example term that is in normal form and a value
  \item Give an example term that is in normal form and not a value (i.e. stuck)
  \item Explain why that example is stuck and cannot be evaluated any further 
\end{itemize}

\subsection*{Induction on Derivation Trees}
Take the definition of $\mathtt{size}$ from the last homework:
\[
\begin{array}{l c r}
\mathtt{size}(n) & \coloneq&  1\\
\mathtt{size}(t_1+t_2) &\coloneq& \mathtt{size}(t_1)+\mathtt{size}(t_2) + 1\\
\mathtt{size}(t_1-t_2) &\coloneq& \mathtt{size}(t_1)+\mathtt{size}(t_2) + 1\\
\mathtt{size}(t_1*t_2) &\coloneq& \mathtt{size}(t_1)+\mathtt{size}(t_2) + 1\\
\mathtt{size}(\mathtt{ifz(t_1;t_2,t_3)}) &\coloneq& \mathtt{size}(t_1)+\mathtt{size}(t_2)+\mathtt{size}(t_3) + 1
\end{array}
\]

Using induction on derivation trees, show that for any two terms $t,t^{\prime}$ with $t\mapsto t^{\prime}$ we have $\mathtt{size}(t) \geq \mathtt{size}(t^{\prime})$.

\section*{Types and Programming Languages}
Read through chapter 4 of types and programming languages and think of a question about this or the previous chapter.
Post your question in the forum thread for the current homework.
Since not all questions from last week were discussed in the tutorial session, you can also repost these questions if they are still open.
