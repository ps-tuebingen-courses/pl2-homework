\section{Simply Typed Lambda Calculus}
We consider the simply typed lambda calculus from the lecture, extended with base types $\mathtt{A}$, $\mathtt{B}$, and $\mathtt{C}$.
Show that that the following terms are well typed in the given contexts by drawing a derivation tree for the typing relation:

\begin{itemize}
  \item $y:\mathtt{A} \vdash ((\lambda{x:\mathtt{A}}.x)\ y) : \mathtt{A}$
  \item $x:\mathtt{B}, y:\mathtt{C} \vdash ((\lambda{z:\mathtt{B}}.\lambda{w:\mathtt{A}}.y)\ x) : \mathtt{A} \to \mathtt{C}$
\end{itemize}
For which of the following terms $t$ does a context $\Gamma$ and a type $T$ exist,
such that they are well typed (i.e., $\Gamma\vdash t : T$)?
If they exist, please write down \(\Gamma\) and \(T\). If not, a short note is enough.
\begin{itemize}
  \item $\lambda{x:\mathtt{B}}.\lambda{y:\mathtt{C}}.(\lambda{z:\mathtt{A}.z})~(w~x~y)$
  \item $\lambda{y:\mathtt{A}}.x~x$
\end{itemize}

\section{Extension with Numbers}
In this task, we extend STLC with numbers.
To the STLC given in TAPL (without booleans (fig. 9.1)), we add the following syntax and evaluation and typing rules:\\
\begin{minipage}{0.45\textwidth}
  \[ 
  \begin{array}{lccr}
    t  & \coloneq  &                  &\emph{terms}\\
    &           & {x}              &\emph{variable}\\
    &           & \lambda{x:T}.t   &\emph{abstraction}\\
    &           & {t~t}            &\emph{application}\\
    &           & \mathtt{zero}    &\emph{constant zero}\\
    &           & \mathtt{succ}~t  &\emph{successor}\\ 
    v  & \coloneq  &                  &\emph{values} \\
    &           & \lambda{x}.t     &\emph{abstraction value}\\
    &           & nv               &\emph{numeric value}\\
  \end{array}
  \]
\end{minipage}
\hfill
\begin{minipage}{0.45\textwidth}
  \[
    \begin{array}{lccr}
      nv & \coloneq  &                  &\emph{numeric values}\\
      &           & \mathtt{zero}    &\emph{zero value}\\
      &           & \mathtt{succ}~nv &\emph{successor value}\\
      \\
      T  & \coloneq  &                  &\emph{types}\\
      &           & T \rightarrow T  &\emph{function type} \\
      &           & \mathtt{Nat}     &\emph{type of natural numbers} \\
    \end{array}
  \]
\end{minipage}
\hfill
\vspace{1em}

\begin{minipage}{\textwidth}
  \begin{minipage}{0.45\textwidth}
    \begin{prooftree}
      \AxiomC{$t_1\mapsto t_1^{\prime}$}
      \RightLabel{\textsc{E-App1}}
      \UnaryInfC{$t_1\ t_2 \mapsto t_1^{\prime}\ t_2$}
    \end{prooftree}
  \end{minipage}
  \hfill
  \vspace{1em}
  \begin{minipage}{0.45\textwidth}
    \begin{prooftree}
      \AxiomC{$t_2\mapsto t_2^{\prime}$}
      \RightLabel{\textsc{E-App2}}
      \UnaryInfC{$v_1\ t_2\mapsto v_1\ t_2^{\prime}$}
    \end{prooftree}
  \end{minipage}
\end{minipage}
\hfill
\vspace{1em}
\begin{minipage}{\textwidth}
  \begin{minipage}{0.45\textwidth}
    \begin{prooftree}
      \AxiomC{\quad}
      \RightLabel{\textsc{E-AppAbs}}
      \UnaryInfC{$(\lambda x:T.t_{12})\ v_2\mapsto t_{12}[x/v_2]$}
    \end{prooftree}
  \end{minipage}
  \hfill
  \begin{minipage}{0.45\textwidth}
    \begin{prooftree}
      \AxiomC{$t\mapsto t^{\prime}$}
      \RightLabel{\textsc{E-Succ}}
      \UnaryInfC{$\mathtt{succ}\ t \mapsto \mathtt{succ}\ t^{\prime}$}
    \end{prooftree}
  \end{minipage}
\end{minipage}
\hfill
\vspace{2em}

\begin{minipage}{\textwidth}
  \begin{minipage}{0.2\textwidth}
    \begin{prooftree}
      \AxiomC{$x:T\in\Gamma$}
      \RightLabel{\textsc{T-Var}}
      \UnaryInfC{$\Gamma\vdash x:T$}
    \end{prooftree}
  \end{minipage}
  \hfill
  \begin{minipage}{0.35\textwidth}
    \begin{prooftree}
      \AxiomC{\quad}
      \RightLabel{\textsc{T-Zero}}
      \UnaryInfC{$\Gamma\vdash\mathtt{zero}:\mathtt{Nat}$}
    \end{prooftree}
  \end{minipage}
  \begin{minipage}{0.35\textwidth}
    \begin{prooftree}
      \AxiomC{$\Gamma,x:T_1 \vdash t_2: T_2$}
      \RightLabel{\textsc{T-Abs}}
      \UnaryInfC{$\Gamma\vdash \lambda x:T_1. t_2 : T_1\rightarrow T_2$}
    \end{prooftree}
  \end{minipage}
  \hfill
\end{minipage}
\hfill
\vspace{1em}

\begin{minipage}{\textwidth}
  \begin{minipage}{0.5\textwidth}
    \begin{prooftree}
      \AxiomC{$\Gamma \vdash t_1:T_{11}\rightarrow T_{12}$}
      \AxiomC{$\Gamma\vdash t_2:T_{11}$}
      \RightLabel{\textsc{T-App}}
      \BinaryInfC{$\Gamma\vdash t_1\ t_2:T_{12}$}
    \end{prooftree}
  \end{minipage}
  \hfill
  \begin{minipage}{0.4\textwidth}
    \begin{prooftree}
      \AxiomC{$t:\mathtt{Nat}$}
      \RightLabel{\textsc{T-Succ}}
      \UnaryInfC{$\Gamma\vdash \mathtt{succ}\ t:\mathtt{Nat}$}
    \end{prooftree}
  \end{minipage}
  \hfill
\end{minipage}
\hfill

\begin{minipage}{\textwidth}
  \vspace{2em}
  \[
    \begin{array}{lcrr }
      x[s/x] & \coloneq & s\\
      y[s/x] & \coloneq & y & \text{ if } x\neq y\\
      (\lambda y:T.t_1)[s/x] & \coloneq & \lambda y:T. t_1[s/x] & \text{ if } y\neq x\\
      (\lambda y:T.t_1)[s/x] & \coloneq & \lambda y:T.t_1 & \text{ if } y=x \text{ of } y\in FV(s)\\
      (t_1\ t_2)[s/x] & \coloneq & t_1[s/x]\ t_2[s/x]\\
      \mathtt{zero}[s/x] & \coloneq & \mathtt{zero}\\
      (\mathtt{succ}\ t)[s/x] & \coloneq & \mathtt{succ}\ t[s/x]
    \end{array}
  \]
  \hfill \vspace{1em}
\end{minipage}
Note that we use the notation $t[s/x]$ instead of $[x\mapsto s]t$ used in the book.\\
Extend the following proofs form Tapl to this extended language

\begin{theorem}\emph{Preservation of Types under Substitution (9.3.8)}\\
  If $\Gamma, x:S\vdash t:T$ and $\Gamma \vdash s:S$, then $\Gamma t[s/x]:T$.
\end{theorem}
\begin{theorem}\emph{Progress (9.3.5)}\\
  Let $t$ be a closed, well-typed term, that is $\vdash t:T$ for some $T$, then either $t$ is a value or there is some $t^{\prime}$ with $t\mapsto t^{\prime}$
\end{theorem}
\begin{theorem}\emph{Preservation (9.3.9)}\\
  If $\Gamma\vdash t:T$ and $t\mapsto t^{\prime}$, then $\Gamma \vdash t^{\prime}:T$.
\end{theorem}
You do not need to repeat the parts of the proof that are identical to the unmodified STLC.
Whenever some part of the proof is identical, note why that is the case.

\section{Environments and free variables}
We will now consider some properties of the typing environments in STLC w.r.t. free variables.\\
Show the following using induction on the typing derivation.
\begin{enumerate}
  \item Prove: If $\Gamma \vdash t : T$ and $x \notin \mathrm{dom}(\Gamma)$, then for all values $v$ we have $t[v/x] = t$.
  \item Prove: If $\Gamma \vdash t : T$, then $FV(t) \subseteq \mathrm{dom}(\Gamma)$.
\end{enumerate}

\section{Compilation to untyped lambda calculus}

\begin{minipage}{0.45\textwidth}
  \[
    \begin{array}{lccr}
      t &\coloneq & \cdots &\emph{terms} \\
      & &\mathtt{true} & \emph{constant true} \\
      & & \mathtt{false} & \emph{constant false} \\
      & & \mathtt{if}~t~\mathtt{then}~t~\mathtt{else}~t & \emph{condition} \\
    \end{array}
  \]
\end{minipage}
\hfill
\begin{minipage}{0.45\textwidth}
  \[
    \begin{array}{lccr}
      \\
      v &\coloneq&  \cdots &\emph{values} \\
      && \mathtt{true} & \emph{true value} \\
      &&\mathtt{false} & \emph{false value} \\
      \\
      T &\coloneq&  \cdots &\emph{types} \\
      & &  \mathtt{Bool} & \emph{type of booleans}
    \end{array}
  \]
\end{minipage}
\hfill
\vspace{2em}

\begin{minipage}{\textwidth}
  \begin{prooftree}
    \AxiomC{$t_1\mapsto t_1^{\prime}$}
    \RightLabel{\textsc{E-If}}
    \UnaryInfC{$\mathtt{if}\ t_1\ \mathtt{then}\ t_2\ \mathtt{else}\ t_3 \mapsto
    \mathtt{if}\ t_1\ \mathtt{then}\ t_2\ \mathtt{else}\ t_3$}
  \end{prooftree}
  \hfill
\end{minipage}
\begin{minipage}{\textwidth}
  \begin{prooftree}
    \AxiomC{\quad}
    \RightLabel{\textsc{E-IfTrue}}
    \UnaryInfC{$\mathtt{if}\ \mathtt{true}\ \mathtt{then}\ t_2\ \mathtt{else}\ t_3 \mapsto t_2$}
  \end{prooftree}
  \hfill
\end{minipage}
\begin{minipage}{\textwidth}
  \begin{prooftree}
    \AxiomC{\quad}
    \RightLabel{\textsc{E-IfFalse}}
    \UnaryInfC{$\mathtt{if}\ \mathtt{false}\ \mathtt{then}\ t_2\ \mathtt{else}\ t_3 \mapsto t_3$}
  \end{prooftree}
  \hfill
\end{minipage}
\vspace{1em}
\begin{minipage}{\textwidth}
  \begin{minipage}{0.45\textwidth}
    \begin{prooftree}
      \AxiomC{\quad}
      \RightLabel{\textsc{T-True}}
      \UnaryInfC{$\Gamma\vdash \mathtt{true} : \mathtt{Bool}$}
    \end{prooftree}
  \end{minipage}
  \hfill
  \begin{minipage}{0.45\textwidth}
    \begin{prooftree}
      \AxiomC{\quad}
      \RightLabel{\textsc{T-False}}
      \UnaryInfC{$\Gamma\vdash \mathtt{false}:\mathtt{Bool}$}
    \end{prooftree}
  \end{minipage}
  \hfill
  \vspace{1em}
  \begin{minipage}{\textwidth}
    \begin{prooftree}
      \AxiomC{$\Gamma\vdash t_1:\mathtt{Bool}$}
      \AxiomC{$\Gamma\vdash t_2:T$}
      \AxiomC{$\Gamma\vdash t_3:T$}
      \RightLabel{\textsc{T-If}}
      \TrinaryInfC{$\Gamma\vdash \mathtt{if}\ t_1\ \mathtt{then}\ t_2\ \mathtt{else}\ t_3 :T$}
    \end{prooftree}
  \end{minipage}
  \hfill
  \vspace{2em}
\end{minipage}
We can extend the $\mathtt{erase}$ function defined in TAPL (Definition~9.5.1) to a new function $\llbracket\cdot\rrbracket$
that translates terms of the extended calculus to pure untyped lambda calculus: 
\begin{align*}
  \llbracket x \rrbracket &\coloneq & x \\
  \llbracket t_1~t_2 \rrbracket &\coloneq&  \llbracket t_1 \rrbracket~\llbracket t_2 \rrbracket \\
  \llbracket \lambda x:T . t \rrbracket &\coloneq& \lambda x. \llbracket t \rrbracket \\
  \llbracket \mathtt{true}\rrbracket &\coloneq& \lambda{x}.\lambda{y}.x \\
  \llbracket \mathtt{false}\rrbracket &\coloneq& \lambda{x}.\lambda{y}.y \\
  \llbracket \mathtt{if}~t_1~\mathtt{then}~t_2~\mathtt{else}~t_3 \rrbracket &\coloneq& (\llbracket t_1 \rrbracket~(\lambda{x}.\llbracket t_2 \rrbracket)~(\lambda{x}.\llbracket t_3 \rrbracket))~(\lambda{y}.y)
\end{align*}
\emph{(In Latex, $\llbracket$ and $\rrbracket$ are written as \textbackslash llbracket and \textbackslash rrbracket. 
This might require you to import and/or install the stmaryrd package)}

\begin{itemize}
  \item Prove: $\llbracket v \rrbracket$ is a value.
  \item Prove: If $t \mapsto^* v$ in the calculus above, then $\llbracket t\rrbracket \mapsto \llbracket v\rrbracket$
      in the untyped call-by-value lambda calculus.
\end{itemize}
Here, $\mapsto^*$ refers to multi-step evaluation.
You can assume $\llbracket t_1 \rrbracket [\llbracket t_2\rrbracket /x] = \llbracket t_1[t_2/x]\rrbracket$.

\section{Types and Programming Languages}
Read chapters 8-10 of Types and Programming languages and post your questions in the dedicated forum thread.
